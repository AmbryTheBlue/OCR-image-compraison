\newglossaryentry{atd}{name={atd.},description={a tak dále}}
\newglossaryentry{tj}{name={tj.},description={to jest}}
\newglossaryentry{en}{name={angl.}, description={anglicky, anglický}}
\newglossaryentry{s}{name={s.}, description={také "\gls{str}", strana(/y), na straně(/ách)}}
\newglossaryentry{str}{name={str.}, description={také "\gls{s}", strana(/y), na straně(/ách)}}
\newglossaryentry{k}{name={k.}, description={kapitola, v kapitole}}
\newglossaryentry{zkr}{name={zkr.}, description={zkratka, zkráceně, zkráceno}}
\newglossaryentry{tzn}{name={tzn.}, description={to znamená}}
\newglossaryentry{tzv}{name={tzv.},description={takzvaný, takzvané}}
\newglossaryentry{napr}{name={např.}, description={například}}
\newglossaryentry{viz}{name={viz}, description={podívej se (viz + nominativ/akuzativ); druhá osoba jednotného čísla rozkazovacího způsobu slovesa vidět}}



\newglossaryentry{ANN}{name={ANN}, description={\gls{en} Artificial Neural Network = Umělá neuronová síť, zde je větší důraz na to, že se nejedná o~síť živočišného původu, oproti \gls{NN}. Podrobněji na \doubleref{sec:NN-uvod}}}
\newglossaryentry{NN}{name={NN}, description={\gls{en} Neural Network(s), v češtině Neuronová(/é) síť(ě). Také \gls{ANN}. Jeden z~typů \gls{AI}, název kvůli podobnosti s~živočišnými nervovými soustavami. Podrobněji na \doubleref{sec:NN-uvod}}}
\newglossaryentry{NS}{name={NS}, description={Neuronová(/é) síť(ě). \Gls{en} Neural Network , také \gls{ANN}. Jeden z~typů \gls{AI}, název kvůli podobnosti s~živočišnými nervovými soustavami. Podrobněji na \doubleref{sec:NN-uvod}}}
\newglossaryentry{FFN}{name={FFN}, description={\gls{en} Feed-foorward Network, v češtině Dopředná Neuronová síť, typ \gls{NN}. Více informací na straně \pageref{sec:NN-FFN}}}

\newglossaryentry{GA}{name={GA}, description={Genetické algoritmy (\gls{en} genetic algorithm(s)), algoritmy inspirované přírodní evolucí, \gls{viz} \ref{sec:genetic}}}

\newglossaryentry{utf-8}{name={UTF-8}, description={\gls{zkr} \gls{en} Unicode Transformation Format. Je to způsob kódování znaků Unicode. To je norma, která zahrnuje většinu písem používaných na Zemi}}

\newglossaryentry{B-MOD}{name={B-MOD}, description={\gls{zkr} \gls{en} Brno Mobile OCR Dataset. Jedná se o~velkou sadu obrázků, více \ref{sec:DataSet}}}



\newglossaryentry{AI}{name={AI}, description={\gls{en} Artificial Intelligence, v češtině Umělá Inteligence, definice na \doubleref{sec:AI}}}
\newglossaryentry{OCR}{name={OCR}, description={\gls{en} Optical Character Recognition, v češtině optické rozpoznávání znaků, více na na straně \pageref{sec:OCR}}}